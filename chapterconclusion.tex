\chapter{CONCLUSION/FUTURE WORK}
\label{chp:concl}
In this thesis, we looked at different approaches for spatial explanations for arbitrary observations. We built an approach called Hierarchical Intervention. According to our evaluation, our approach outperforms aggravation and intervention in precision while it outperforms aggravation in recall. If we look at our approach from the context of a data analyst, it is designed to provide an advantage over traditional data analytics using relational database systems. The reason for this is because we use programming paradigms for distributed processing. Even though our implementation is designed to work on Apache Spark, the paradigms used may be useful for future distributed processing systems like GOLEM which have an even larger processing scope \citep{golem2018}. In addition to saving valuable time for data analyst, our system is also useful for reducing the scope of work for the analyst. The proposed system can be used as a search engine much like Google where the analyst uses observations instead of search terms and the system comes up with explanations instead of web pages. Instead of looking at the entire data, the analyst can look at a small subset represented by these explanations to find what they are looking for.

There are a number of improvements that can be made on top of our proposed approach that we weren't able to implement due to the lack of time or availability of datasets with the ground truth. One interesting idea is to use influence and intensity as parameters of perceptrons in a neural network\citep{grossberg1988nonlinear,widrow199030}. The neural network can be trained to explain specific datasets by using ground truth.
Many spatial datasets that we find also have a temporal component. Our system does not handle the temporal aspect of the data. An improvement that can be made to our approach is to use time range as a secondary component of our candidate explanations. We can build a temporal hierarchy in the same way we build a spatial hierarchy. The spatiotemporal hierarchy can be represented as a pyramid instead of a tree.
If we extend our idea of the spatiotemporal system to a neural network, we can think of it from the perspective of a recurrent neural network\citep{chung2016hierarchical}. It might also help to use Long Short Term Memory(LSTM) model to encapsulate the value of explanations at levels of the hierarchy which are far apart \citep{hochreiter1997long}.
Since influence and intensity are expensive operations, the time complexity of the network should also be taken into account and heuristics should be used where necessary when using approaches with a lot of inputs.
